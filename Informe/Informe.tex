%CONFIG DEL DOC
\documentclass[a4paper, 12pt, spanish]{article}
\usepackage[utf8x]{inputenc}
\usepackage{babel}
\usepackage{multicol, enumitem}
\usepackage{graphicx}
\graphicspath{ {images/} }
\usepackage{anysize} 

\marginsize{2.5cm}{2.5cm}{2cm}{2cm} 


\def\code#1{\texttt{#1}}


\title{Informe Trabajo Práctico N°2 \linebreak \textit{Atari Asteroids} \linebreak}

\author{Malena Díaz Falvo y Agustín Ormazábal}
\date{30 de Junio de 2019}

\begin{document}


\begin{figure}[t]
	\includegraphics[width=8cm]{logoFIUBA} 
\end{figure}
\maketitle

\thispagestyle{empty}
\newpage

\setcounter{page}{2}
\pagestyle{plain}

\section*{Introducción}
El actual proyecto consta de la implementación de una reversión del juego
Asteroids en lenguaje ISO-C99 utilizando la biblioteca gráfica SDL2. Para su recreación
se implementó el concepto de TDA, entre otras funcionalidades, incorporados a lo largo
del curso. \newline
 
\section*{Desarrollo}
Para abordar este proyecto, se utilizaron y modificaron algunos de los archivos brindados para el TP1 (reversión del Lunar 
Lander). Además, se diseñaron y utilizaron diferentes TDAs para los objetos del juego, obteniendo finalmente los siguientes \texttt{.h}:

\begin{multicols}{2}
\begin{itemize}[label=$\bullet$]

	\item \texttt{caracteres.h}
	\item \texttt{config.h} 
	\item \texttt{dibujar.h}
	\item \texttt{diccionario.h}
	\item \texttt{movimiento.h}
	\item \texttt{nave.h}
	\item \texttt{asteroides.h}
	\item \texttt{disparos.h}
	\item \texttt{graficador.h}
	\item \texttt{vector.h}
	\item \texttt{iterador.h}
	\item \texttt{listas.h}
	

\end{itemize}
\end{multicols}
\medskip

[cambiar a partir de acá]
Cada uno de estos archivos está asociado a su correspondiente \texttt{.c} (a excepción de \texttt{config.h}), 
tres de los cuales fueron suministrados por la cátedra, y cada uno posee una descripción detallada
de las pre y postcondiciones de cada función junto con, valga la redundancia, la función de las mismas. Además, se diseñó un
diccionario para traducir caracteres a los vectores que se encuentran en \texttt{caracteres.h} con el fin
de poder utilizar la librería gráfica a nuestra voluntad. \newline

Al inicio de cada partida, la cual se ve delimitada por el aterrizaje o destrucción de la nave, 
se genera un terreno aleatorio nuevo, para esto se utilizó la función
\texttt{float **terreno$\_$crear(size$\_$t *n);} brindada en el enunciado del trabajo práctico. 
Además, se posicionarion los mensajes con sus respectivos valores, que se modificarán a lo largo del
transcurso del juego, de forma tal que se asemeje al diseño original del mismo.\newline


\section*{Dificultades}
Personalmente, la mayor dificultad que se me presentó fue comenzar a utilizar la función \texttt{SDL$\_$RenderDrawLine(renderer, x0, y0, x1, y1);}
de la biblioteca SLD2 y llegar a la conclusión de que era conveniente utilizar vectores dinámicos, para representar a la nave y al chorro,
que sean redefinidos en cada ciclo de \texttt{while}. Esto es necesario debido al hecho de que la nave debe rotar sobre sí
misma, dependiendo de la interacción asincrónica que se establece con el o la usuaria, mientras se traslada por el plano xy. Si, por el
contrario, los vectores se definen por fuera del \texttt{while}, cada traslación los modificaría y se perdería el rastro del orígen de coordenadas
para las figuras, por lo que terminarían describiendo una circunferencia a lo largo del plano en lugar de rotar sobre sí
mismos.\newline

Por otro lado, la modularización contribuyó bastante a tener un código ordenado y prolijo. Sin embargo, si no se lo actualiza con cuidado
regularmente, se puede llegar a olvidar sacar algún prototipo de una función borrada de los módulos. Esto no implica un error de 
compilación pero podría generar cierta confusión para aquel o aquella que quiera utilizar nuestros \texttt{.h} como una librería. Además
al ser un enunciado de gran extención, a pesar de la lectura y relectura del mismo, siempre parecía que algun dato pasaba desapercibido.\newline


\section*{Adicional}
En la versión original del Lunar Lander se hace un zoom al momento de aterrizar. Con esta base, y teniendo dos diseños
para la figura (nave chica y nave grande), se definió una distancia mínima al terreno y un intervalo del ángulo de la nave,
y dependiendo de si se cumplen o no dichas condiciones, se dibujarán en pantalla la \textbf{nave de aterrizaje} (nave grande) 
o \textbf{la nave de vuelo} (nave chica). \bigskip

%\begin{figure}[h]
% 	\centering
%	\includegraphics[width=8cm]{naves} 
%	\caption{Naves suministradas.}
%\end{figure}

\bigskip
Esta elección se debió a que, a pesar de que ambas naves poseen sus patas desplegadas para el aterrizaje,
la primera posee un aspecto más sólido, con una mayor resistencia ante el impacto de la nave
contra la superficie (en caso de que éste sea un aterrizaje violento), mientras que la segunda parece más frágil e inestable.
\newline

Además, se le agregaron estrellas al fondo para recrear el escenario de la superficie lunar con su paisaje de la forma más
acertada posible. Para esto se definió una matriz de coordenadas para la representación de una estrella, junto con las funciones necesarias
para crear una cantidad determinada de manera aleatoria e imprimirlas.

%\begin{figure}[h!]
% 	\centering
%	\includegraphics[width=16cm]{estrellas} 
%	\caption{Superficie lunar en el juego.}
%\end{figure}
%\bigskip

\newpage
Al no ser  modificaciones complicadas de implementar, que aportan a la estética del juego, creí conveniente 
su aplicación ya que no interfiere con los campos principales a ser evaluados ni modifica los demás requisitos
exigidos en el enunciado del trabajo práctico.



\end{document}